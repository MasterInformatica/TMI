\section{El tablero del juego}
\label{s3:sec:tablero}

\subsection{Ficheros xml}
\label{s3:subsec:xml}
Para la implementaci�n de los niveles se ha decidido crear una estructura
xml que permita guardar cada nivel en un fichero independiente. De esta
manera, solamente es necesaria una escena que carga un mapa u otro
dependiendo del nivel elegido.

La estructura del fichero xml elegida consta de 3 partes bien
diferenciadas:
\begin{enumerate}
  \item \texttt{dimensions}: Representa el tama�o que va a tener cada celda
    del juego, las cubiertas, el tama�o del mapa y la escala respecto al
    prefab guardado.
  \item \texttt{board}: Representa a las casillas del tablero. Permite
    establecer una altura por defecto a cada celda. Adem�s, cada celda
    viene definida por las coordenadas (x,z), la altura y, y un atributo
    indicando si es una celda objetivo o no (\emph{goal}).
  \item \texttt{robot}: Indica la posici�n original del robot, y la
    rotaci�n con la que tiene que comenzar.
\end{enumerate}

En el c�digo~\ref{s3:lst:xml} se muestra un ejemplo de nivel, que
corresponde al nivel 1 mostrado en la figura~\ref{s1:fig:interfaz}.\\

\begin{lstlisting}[caption={Ejemplo de nivel 1 del juego}, label=s3:lst:xml]
<xml>
	<dimensions>
		<board length_x="10" length_z="10" scale="2"/>
		<block size="1" separation="0.1" />
		<cover size="0.1" />
		<robot height="2" />
	</dimensions>

	<board default_height="0" >	
		<tile x="2" z="0" height="1" />
		<tile x="2" z="1" height="1" />
		<tile x="2" z="2" height="2" />
		<tile x="1" z="2" height="1" />
		<tile x="0" z="2" height="2" />
		<tile x="0" z="1" height="1" />
		<tile x="0" z="0" height="2" attr="goal"/>	
	</board>
	
	<robot>
		<position x="2" y="1" z="0" />
		<rotation x="0" y="-90" z="0" />
	</robot>
</xml>
\end{lstlisting}

\subsection{Leyendo el fichero, creando el tablero}
Para la lectura del nivel en formato xml, y la creaci�n del tablero por
pantalla, se utiliza el script de nombre \texttt{GUI\_Layout.cs}, que
realiza ambas cosas. Para la creaci�n por pantalla, se utilizan prefabs que
representan a los bloques y cubiertas de los bloques, as� como los
distintos materiales usados para encender y apagas las celdas. Estos
elementos se asocian al script mediante la interfaz de unity.

Para almacenar las casillas internamente, se ha decidido almacenar de
manera separada la definici�n de la representaci�n visual. Para la
definici�n de la celda, se utilizan las siguientes estructuras, que m�s
tarde se utilizar�n desde el script:

\begin{minipage}{0.5\textwidth}
  \begin{lstlisting}
/* Definicion de una casilla. */
public class TileDef {
  public TileType type;
  private int _height;
  public int height {
    get{ return _height; }
    set{
      _height = value;
      if (value <= 0)
      type = TileType.none;
    }
  }
  
  public bool isOn;
  public int x;
  public int z;
}
  \end{lstlisting}
\end{minipage}
\begin{minipage}{0.5\textwidth}
  \begin{lstlisting}
/* Tipos para una casilla */
public enum TileType{
  normal,
  goal,
  none
}
  \end{lstlisting}
\end{minipage}


El script lo que realiza en primer lugar es la lectura del tablero xml
sobre las variables. Una vez le�do el tablero entero, se dibuja la interfaz
a partir de estas variables, y en �ltimo lugar se coloca al robot en su
posici�n correcta.

El m�todo que lee el fichero xml es el siguiente:
\begin{lstlisting}[caption={M�todo encargado de la lectura del fichero
      xml}]
/** Script encargado de toda la interfaz de la parte izquierda.
 * Crea el tablero a partir de un fichero xml, y coloca al robot en la posicion adecuada.
 * Almacena el tablero de forma logica para poder comprobar si los movimientos son posibles.
 * La representacion fisica de la representacion logica se encuentran almacenadas de manera independiente.
 */
public class GUI_Layout : MonoBehaviour {
  public static GUI_Layout S;//Singleton para acceder
  
  // Prfebs para crear una casilla
  public GameObject tileBlock_prefab;
  public GameObject specialCover_prefab;
  public GameObject normalCover_prefab;
  
  // Materiales para el cover de una casilla
  public Material offCover_material;
  public Material onCover_material;
  
  //Prefab del robot
  public GameObject robot_prefab;
  
  
  // Lectura del xml
  public PT_XMLReader      xmlr;  // PT_XMLReader
  public PT_XMLHashtable   xml;
  
  
  // Lista de xml para cada nivel
  public TextAsset[] niveles;
  
  
  // tamanyo maximo del tablero
  public int MAX_SIZE=20;
  
  //Texto para mostrar el numero de nivel
  public GameObject LevelTxt;
  
  
  public bool _____________________ ;
  public TextAsset texto;
  
  // Datos para pintar el mapa. Leidos del xml
  private float blockSize = 1f; 
  private float blockSeparation = 0.1f;
  private float coverSize = 0.1f; 
  private float scale = 2f;
  
  //Datos para pintar el robot. Leidos del xml
  private Vector3 robotPosition;
  private Vector3 robotRotation;
  private float robotHeight;
  
  //tamanyo del tablero
  private int size_x;
  private int size_z;
  private int default_height;
  
  // Definiciones e instancias de las celdas
  public TileDef[,] board_def;
  public GameObject[,] board;
  
  //Anchor para el board
  public GameObject boardAnchor;
  
  //Gameobject del robot para poder destruirlo
  public GameObject robot;
  
  
  
  public void Awake(){
    S = this;
    this.texto = this.niveles[levelLoad.level-1];
    
    GUIText gt = this.LevelTxt.GetComponent<GUIText>();
    gt.text = "Nivel: " + levelLoad.level;
  }
  
  
  public void Start(){
    this.resetLayout ();
    
    this.readLayout (texto.text);
    
    this.drawlayout ();
    this.drawRobot();
    
  }
  
  
  // -------------------------------------
  // -------------- METODOS --------------
  // -------------------------------------
  
  /*
   * carga el tablero definido en el xml pasado por paramtero
   */
  public void readLayout(string xmlText){
    //Antes de hacer nada, reseteamos el tablero
    this.resetLayout ();
    
    xmlr = new PT_XMLReader();
    xmlr.Parse(xmlText); 
    xml = xmlr.xml["xml"][0];
    
  
    //1.- Definiciones de las dimensiones
    PT_XMLHashtable dimensions = xml["dimensions"][0];
    //1.1 - tablero
    this.size_x = int.Parse (dimensions ["board"][0].att ("length_x"));
    this.size_z = int.Parse (dimensions ["board"][0].att ("length_z"));
    this.scale = float.Parse (dimensions ["board"][0].att ("scale"));
    
    //1.2 - bloque
    this.blockSize = float.Parse (dimensions ["block"][0].att ("size"));
    this.blockSeparation = float.Parse (dimensions ["block"][0].att ("separation"));
    
    //1.3 - cover
    this.coverSize = float.Parse (dimensions ["cover"][0].att ("size"));
    
    //1.4 - robot
    this.robotHeight = float.Parse (dimensions ["robot"][0].att("height"));
    
    
    //2.- Celdas por defecto
    this.default_height = int.Parse (xml ["board"] [0].att ("default_height"));
    this.initBoard (this.default_height, this.size_x, this.size_z);
    
    
    //3.- Definimos de las celdas
    PT_XMLHashList tileX = xml["board"][0]["tile"];
    for (int i=0; i<tileX.Count; i++) {
      int x, z;
      x = int.Parse (tileX[i].att("x"));
      z = int.Parse (tileX[i].att("z"));
      
      
      this.board_def[x, z].height = int.Parse(tileX[i].att ("height"));
      
      string attr = (tileX[i].HasAtt("attr")) ? tileX[i].att ("attr") : "none";
      switch (attr){
        case "goal":
          this.board_def[x,z].type=TileType.goal;
          Director.S.totalGoals ++;
          break;
        default:
          this.board_def[x,z].type = TileType.normal;
          break;
      }
      this.board_def[x,z].isOn = false;
    }
    
    //4.- Posicion inicial del robot.
    PT_XMLHashtable robot = xml["robot"][0]["position"][0];
    this.robotPosition = new Vector3(float.Parse(robot.att("x")),
                float.Parse(robot.att("y")),
                float.Parse(robot.att("z")));
    robot = xml["robot"][0]["rotation"][0];
    this.robotRotation = new Vector3(float.Parse(robot.att("x")),
                float.Parse(robot.att("y")),
                float.Parse(robot.att("z")));
    
    //Transformamos la rotacion en el intervalo [0,360)
    while(this.robotRotation.y < 0){
      this.robotRotation.y += 360;
    }
    while(this.robotRotation.y >=360){
      this.robotRotation.y -= 360;
    }
}
\end{lstlisting}


Una vez le�do los datos del tablero, se dibuja la interfaz asociada a este
tablero, y tambi�n se coloca el robot en su posici�n. Los m�todos
encargados de esto son los siguientes:

\begin{lstlisting}[caption={M�todos para dibujar el tablero y al robot}]
/** Dibuja los bloques del tablero segun la definicion guardada en this.board_def.
* Agrupa todos los elementos en un layout. Se supone que se ha invocado
* con anterioridad a resetLayout() que borra todos los hijos del anchor
* (es decir, borra todos los bloque creados)
*/
private void drawlayout(){
  GameObject tile;
  for (int i=0; i<this.size_x; i++) {
    for (int j=0; j<this.size_z; j++) {
      if(this.board_def[i,j].type == TileType.none)
      continue;
      
      //Creamos el tile
      tile = createTile(this.board_def[i,j].height, this.board_def[i,j].type == TileType.goal);
      tile.name = "" + i + "x" + j;
      //posicion segun la i, j
      tile.transform.position = new Vector3(i*(2*this.blockSize + this.blockSeparation), 0,
      j*(2*this.blockSize + this.blockSeparation));
      //hijo del anchor
      tile.transform.parent = this.boardAnchor.transform;
      
      //lo guardamos en el mapa
      this.board[i,j] = tile;
    }
  }
}

/** 
* Dibuja al robot en la posicion original leida del xml.
* El robot viene definido por una posicion inicial, y una direccion
* a la que se encuentra mirando.
*/
private void drawRobot(){
  GameObject robot = Instantiate(this.robot_prefab) as GameObject;
  
  //Colocar en la posicion y la rotacion
  robot.transform.position = new Vector3(this.robotPosition.x*(2*this.blockSize + this.blockSeparation),
  (this.robotPosition.y-1)*(this.blockSize + this.blockSeparation) + 
  this.robotHeight + this.blockSize/2 + this.coverSize,
  this.robotPosition.z*(2*this.blockSize + this.blockSeparation));
  
  robot.transform.rotation = Quaternion.Euler(this.robotRotation);
  
  //Calculamos la direccion a la que mira
  Vector3 dir;
  if(this.robotRotation.y < 90)
    dir = new Vector3(1,0,0);
  else if (this.robotRotation.y < 180)
    dir = new Vector3(0,0,-1);
  else if (this.robotRotation.y < 270)
    dir = new Vector3(-1,0,0);
  else 
    dir = new Vector3(0,0,1);
  
    ((Movement)robot.GetComponent<Movement>()).dirAct = dir;
    ((Movement)robot.GetComponent<Movement>()).rotAct = this.robotRotation;
    
    
    ((Robot)robot.GetComponent<Robot> ()).init ((int)(this.robotPosition.x), (int)(this.robotPosition.y), 
    (int)(this.robotPosition.z),
    (int)dir.x, (int)dir.z);
    
    this.robot = robot;
}


/*
* Resetea las variables necesarias para iniciar una nueva ejecucion.
*/
private void resetLayout(){
  this.board_def = new TileDef[MAX_SIZE, MAX_SIZE];
  if(this.board != null){
    for (int i=0; i<MAX_SIZE; i++)
    for (int j=0; j<MAX_SIZE; j++)
    Destroy(this.board[i,j]);
  }
  
  this.board = new GameObject[MAX_SIZE, MAX_SIZE];
  
  //Destruimos y volvemos a crear el boardAnchor
  if (this.boardAnchor != null) {
    Destroy (this.boardAnchor);
  }
  this.boardAnchor = new GameObject ();
  this.boardAnchor.name = "BoardAnchor";
  
  if(this.robot != null)
  Destroy(robot);
  
  Director.S.totalGoals = 0;
}


/*
 * Inicia el tablero de tamanyo nxm con celdas de altura h.
 * Si es una casilla fuera de n,m lo inicializa con una celda a none.
 */
private void initBoard (int h, int n, int m){
  TileDef td;
  for (int i=0; i<this.MAX_SIZE; i++) {
    for (int j=0; j<this.MAX_SIZE; j++) {
      td = new TileDef();
      td.x = i;
      td.z = j;
      td.height = h;
      
      
      if(i> n || j> m){ //fuera del tablero
	td.type = TileType.none;
	this.board_def[i,j] = td;
      } else { //Celda buena por defecto
	td.type = (h<=0) ? TileType.none : TileType.normal;
	this.board_def[i,j] = td;
      }
    }
  }
}

	
/*
 * Devuelve un objeto con una torre de bloques creada, de altura pasada por parametro.
 * El atributo special indica si es una casilla objetivo o no
 */
private GameObject createTile(int height, bool special){
  GameObject tile = new GameObject ();
  
  //1.- Bloques: escala, hijos del tile, apilar encima.
  GameObject go;
  for (int i = 0; i<height; i++){
    go = Instantiate(this.tileBlock_prefab) as GameObject;
    go.transform.localScale = new Vector3(this.scale, this.blockSize, this.scale);
    go.transform.parent = tile.transform; //hijo del tile
    go.transform.position = new Vector3(0, i*(this.blockSize + this.blockSeparation), 0); //encima del anterior
    go.name = "block_" + i;
  }
  
  //2.- Cover
  go = special ? Instantiate (this.specialCover_prefab) : Instantiate (this.normalCover_prefab);
  go.transform.localScale = new Vector3 (this.scale, this.coverSize, this.scale);
  go.transform.parent = tile.transform;
  
  float h = (height - 1) * (this.blockSize + this.blockSeparation) + this.blockSize / 2f + this.coverSize / 2f; 
  go.transform.position = new Vector3(0, h, 0);
  go.name = "cover";
  
  return tile;
}

\end{lstlisting}

Por �ltimo, se incorpora un m�todo que permite cambiar el color de una
celda en el tablero:
\begin{lstlisting}[caption={M�todo para cambiar el color de una celda
      especial}]
/*
 * Ilumina la celda de la possicion (x,z) si esta es de tipo goal.
 * Si estaba encendida, la apaga.
 */
public int SwitchLight(int x, int z){
  if (this.board_def [x, z].type != TileType.goal)
    return 0;
  
  TileDef td = this.board_def[x,z];
  GameObject go = this.board [x, z];
  
  Renderer rmat = null;
  foreach (Transform t in go.transform) {
    if (t.name == "cover"){
      rmat = t.gameObject.GetComponent<Renderer>();
    }
  }
  
  
  if (td.isOn == false) {
    rmat.material = this.onCover_material;
    td.isOn = true;
    
    return 1;
  }
  else{
    rmat.material = this.offCover_material;
    td.isOn = false;
    
    return -1;
  }
}
\end{lstlisting}
