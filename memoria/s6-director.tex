\section{Coordinaci�n entre las dos interfaces: El director de orquesta}
\label{s6:sec:director}

Para la gesti�n entre las dos interfaces (la que graba las acciones del
usuario y el robot que las reproduce), se utiliza un script llamado
\texttt{Director.cs}. Este script realiza m�ltiples acciones, aunque las
m�s importantes son las siguientes:
\begin{itemize}
  \item Almacena las acciones que va introduciendo el usuario, y manda al
    panel correspondiente que la pinte.
  \item Gestiona el panel activo. Manda colorear al panel correspondiente.
  \item Reinicia la interfaz correspondiente cuando el jugador pulsa alg�n
    bot�n que desencadene esta acci�n (stop o basura).
  \item Reproduce los comandos en orden utilizando una estructura de pila.
\end{itemize}

\subsection{Reproduciendo los comandos}
A la hora de reproducir comandos se presentan dos retos que afrontar:
\begin{enumerate}
  \item Hay que mantener la traza de ejecuci�n para poder colorear el
    comando activo y eliminar el anterior.
  \item Tiene que ser v�lido para llamadas recursivas.
\end{enumerate}

Los comandos se encuentran almacenados en tres listas, una para cada
procedimiento (\emph{Main}, \emph{P1} y \emph{P2}). Para reproducirlos en
orden y almacenar la traza se utiliza una pila basada en ciclos. En cada
inicio de ciclo se desapila la acci�n a realizar junto con el panel del que
realizarlo, y al finalizar el ciclo, se apila la nueva acci�n junto con el
panel correspondiente. La gesti�n de esta pila, junto con el resto de
m�todos del script es la siguiente:\\

\begin{lstlisting}[caption={Director.cs}]
using UnityEngine;
using System.Collections;
using System.Collections.Generic;

/**
 * Estructura auxiliar usada para la reproduccion de los comandos
 */
struct PairInt {
  public PairInt(int a, int b) {
    x = a;
    y = b;
  }
  public int x, y;
}


/**
 * Director de orquesta. Es el encargado de llevar el control de todo el juego.
 * Maneja los paneles y controla las acciones pulsadas.
 * Almacena la lista de todas acciones a ejecutar.
 * La reproduccion la reliza mediante un sistema de pilas.
 * Avisa al layout para colorear/descolorear las celdas.
 * Controla la victoria del juego
 */
public class Director : MonoBehaviour {
  public static Director S; //Singleton
  
  public int totalGoals; //Numero de objetivos. Usado para condicion de victoria
  public GameObject texto; //Texto a mostrar cuando se supera un nivel
  
  public bool ______________________________;
  
  public List< List<ActionType> > acciones; //Lista de lista de acciones. Por cada panel existe una lista de acciones.
  public bool recording; //Bool indicando si se esta reproduciendo o grabando acciones
  public int panelActiveId; // Int indicando el panel actual activo
  
  
  private Stack<PairInt> llamadas; //Pila utilizada para realizar la reproduccion de los comandos
  
  
  
  public void Awake(){
    S = this;
  }
  
  
  public void Start(){
    this.recording = true;
    
    this.acciones = new List<List<ActionType>> ();
    this.acciones.Add(new List<ActionType>());
    this.acciones.Add(new List<ActionType>());
    this.acciones.Add(new List<ActionType>());
  }
  
  
  // A�ade la accion pasada por parametro al panel actual.
  // Guarda la accion en la lista de acciones local.
  public void addAction(ActionType at){
    if (!this.recording)
      return;

    this.acciones[this.panelActiveId].Add (at);
    Actions_Layout.S.insertInPanel (this.panelActiveId, at);
  }


  // Ejecuta los comandos que estaban guardados
  // Para ello, apila en una pila parejas <int,int>, que representan 
  // <panelId, ActionId>. Simula una pila de llamadas de funciones.
  public void play(){
    Actions_Layout.S.changeButton();
    
    if (this.recording) {
      this.recording = false;
      this.llamadas = new Stack<PairInt>();
      this.llamadas.Push(new PairInt(0,0));
      Actions_Layout.S.switchOffAllLights();
      
      Next ();
    }else{
      this.restartLevel();
    }
  }
  
  
  // Ejecuta la siguiente accion.
  // Utiliza el sistema de pila para simular las llamadas a funcion.
  // De esta maenra se consiguen llamadas recursivas, y coloreado de la
  // interfaz.
  public void Next(){
    if (this.recording)
      return;
    
    if (this.llamadas.Count > 0) {
      PairInt pi = this.llamadas.Pop ();
      
      if(pi.y < 0){ //si acabamos una rutina, apagamos el ultimo elemento
	Actions_Layout.S.lightCube (pi.x, this.acciones[pi.x].Count-1, false);
	Invoke("Next", 1);
      }else
        this.ejecutaAccion (pi.x, pi.y);
    }
  }
  
  
  //Ejecuta la accion i sobre la lista a, y apila la siguiente en la pila de llamadas
  public void ejecutaAccion(int a, int i){
    
    //apagamos la anterior si podemos
    if (i > 0)
      Actions_Layout.S.lightCube (a, i - 1, false);
    //enciende la actual
    Actions_Layout.S.lightCube (a, i, true);
    
    //apilamos la sigiente ejecucion, si existe
    if (i + 1 < this.acciones [a].Count)
      this.llamadas.Push (new PairInt (a, i + 1));
    else
      this.llamadas.Push (new PairInt (a, -1));


    switch (this.acciones[a][i]) {
      case ActionType.up:
        Robot.S.moveUP ();
	break;
      case ActionType.jump:
        Robot.S.jump();
	break;
      case ActionType.right:
        Robot.S.rotateRight();
	break;
      case ActionType.left:
        Robot.S.rotateLeft();
        break;
      case ActionType.light:
        int ret = GUI_Layout.S.SwitchLight(Robot.S.x,Robot.S.z);
	if (ret > 0) this.totalGoals --; 
	if (ret < 0) this.totalGoals++;

	testGoal();
	Invoke("Next",1);
	break;
      case ActionType.p1:
        this.llamadas.Push(new PairInt(1, 0));
	Invoke("Next", 1);
	break;
      case ActionType.p2:
        this.llamadas.Push (new PairInt(2,0));
  	Invoke("Next", 1);
	break;
      default:
      break;
    }
  }
  

  // Comprueba si se ha alcanzado la condicion de victoria
  public void testGoal(){
    if (this.totalGoals == 0) {
      this.recording = true;
      this.texto.SetActive(true);
      
      levelLoad.levelPassed[levelLoad.level-1] = true;
      Invoke("LoadMenu", 2);
    }
  }
  
  public void LoadMenu(){
    Application.LoadLevel("_Scene_menu");
  }
  
  
  // Es llamado cuando se desea borrar todas las acciones.
  // Vacia la lista local y manda la orden a la interfaz
  public void trash(){
    if (!this.recording)
      return;
      
    for(int i=0; i<this.acciones.Count; i++)
      this.acciones[i].Clear ();
        
      Actions_Layout.S.clearPanels ();
  }

  
  public void restartLevel(){
    this.recording = true;
    
    GUI_Layout.S.restartLayout();
    Actions_Layout.S.switchOffAllLights();
    
    this.panelActiveId = 0;
    Actions_Layout.S.paneles [0].activatePanel ();
  }
  
  
  // Almacena el panel activo y ordena que se pinte al nuevo.
  // Apaga el panel activo viejo
  public void panelActive(int i){
    if(i != this.panelActiveId)
      Actions_Layout.S.paneles[this.panelActiveId].deactivatePanel();
      
      this.panelActiveId = i;     
  }
}
\end{lstlisting}
