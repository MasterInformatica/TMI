\section{El robot}
\label{s4:sec:robot}
El personaje del juego viene representado por la idea de un robot, que se
corresponde a un cilindro que cumple la funci�n de cuerpo y un cubo que
cumple la funci�n de nariz o cara del mismo.

La l�gica del personaje viene implementada a trav�s de dos componentes:
\texttt{Movements.cs} y \texttt{Robot.cs}. El primero de ellos es el
encargado de realizar los movimientos del robot sin consultar si el mapa lo
permite o no, mientras que el segundo es el encargado de comprobar si el
movimiento a realizar es v�lido o no.

\subsection{El movimiento}
\label{s4:subsec:mvto}
Los movimientos que puede realizar el robot son 4: avanzar, girar a la
derecha y a la izquierda, y saltar.

Para avanzar, se utiliza una funci�n auxiliar que calcula la posici�n final,
y se ayuda de variables locales que indican la posici�n actual del robot:

\begin{lstlisting}[caption={Avanzar hacia delante}]
using UnityEngine;
using System.Collections;
using System.Collections.Generic;


/**
 * Clase encargada de realizar los movimientos por el tablero.
 * Realiza los movimientos de avanzar, girarDcha, girarIzq, saltar.
 * 
 * Esta clase no comprueba si el movimiento se puede realizar o no. Tiene
 * que ser complementada con otro componente que se encargue de la logica
 * antes de llamar a los metodos de este componente
 **/
public class Movement : MonoBehaviour {

  //Vectores que indican hacia donde se encuentra mirando el gameObject actualemnte,
  // asi como la rotacion actual. Sirve para poder calcular la direccion y giro de destino.
  public Vector3 dirAct = new Vector3 (0, 0, 1f);
  public Vector3 rotAct = new Vector3 (0, 0, 0);
  
  //Para mover hacia delante, fijamos el vector de desplazamiento, y multiplicamos por
  // la direccion actual para eliminar las coordenadas que no nos interesan.
  public Vector3 dirFrente = new Vector3(2.1f,0,2.1f);
  public Vector3 dirArriba = new Vector3 (0, 1f, 0);
  
  public Vector3 rotLeft = new Vector3 (0, -90f, 0);
  public Vector3 rotRight = new Vector3 (0, 90f, 0);
  
  
  public float MVTO_DURATION = 5f;
  
  public bool _____________________________;
  
  public float timeStart; //Timestamp en el que comienza el movimiento.
  public float timeDuration; //Tiempo que queda para que acabe el mvto.
  
  public bool isMoving; //Inidica que se mueve de frente.
  public bool isRotating;
  public bool isJumping;
  
  public Vector3 poi; //Posicion destino
  public Vector3 orgPosition; //posicion original
  
  public Vector3 rotPoi; //Rotacion final
  public Vector3 orgRotation; //Rotacion original
  
  public List<Vector3>      bezierPts;
  
  
  /** 
  * Comienza el movimiento hacia delante de la entidad.
  * No comprueba si se encucentra ya en movimiento
  **/
  public void Avanza(){
    this.isMoving = true;
    this.isRotating = false;
    this.isJumping = false;
    
    this.timeStart = Time.time;
    this.timeDuration = this.MVTO_DURATION;
    
    
    this.orgPosition = this.transform.position;
    this.poi = this.GetAvanzaPoi ();
  }

  /**
  * Devuelve a partir de la posicion actual y hacia donde este mirando el destino 
  *  de un movimiento en linea recta
  **/
  private Vector3 GetAvanzaPoi(){
    Vector3 aux = new Vector3 (this.dirAct.x * this.dirFrente.x,
    this.dirAct.y * this.dirFrente.y,
    this.dirAct.z * this.dirFrente.z);
    
    return this.transform.position + aux;
  }
}
\end{lstlisting}

Para las rotaciones, se realizan de manera similar. Se a�ade un m�todo que
se encarga de actualizar los atributos de posici�n y rotaci�n de manera
adecuada.


\begin{lstlisting}[caption={Rotaciones hacia la izquierda y derecha.}]
/**
* Rotacion hacia la izquierda
**/
public void RotateLeft(){
  this.isMoving = false;
  this.isRotating = true;
  this.isJumping = false;
  
  this.timeStart = Time.time;
  this.timeDuration = this.MVTO_DURATION;
  
  this.orgRotation = this.rotAct;
  this.rotPoi = this.rotAct + this.rotLeft;
  
  this.ActualizaDireccionTrasRotacion (true);
}

/**
* Rotacion hacia la derecha
**/
public void RotateRight(){
  this.isMoving = false;
  this.isRotating = true;
  this.isJumping = false;
  
  this.timeStart = Time.time;
  this.timeDuration = this.MVTO_DURATION;
  
  this.orgRotation = this.rotAct;
  this.rotPoi = this.rotAct + this.rotRight;
  
  this.ActualizaDireccionTrasRotacion (false);
}


  
/**
* Actualiza el vector de la direccion actual (hacia donde apunta el gameObject),
*  tras un giro. 
* El parametro indica si ha girado hacia la derecha o hacia la izquierda.
**/
private void ActualizaDireccionTrasRotacion(bool left = true){
  Vector3 final = this.dirAct;
  
  if (left) {
    if(final.x == 1 && final.z == 0)
      this.dirAct = new Vector3(0,0,1);
    if(final.x == -1 && final.z == 0)
      this.dirAct = new Vector3(0,0,-1);
    if(final.x == 0 && final.z == 1)
      this.dirAct = new Vector3(-1,0,0);
    if(final.x == 0 && final.z == -1)
      this.dirAct = new Vector3(1,0,0);
  }
  else {
    if(final.x == 1 && final.z == 0)
      this.dirAct = new Vector3(0,0,-1);
    if(final.x == -1 && final.z == 0)
      this.dirAct = new Vector3(0,0,1);
    if(final.x == 0 && final.z == 1)
      this.dirAct = new Vector3(1,0,0);
    if(final.x == 0 && final.z == -1)
      this.dirAct = new Vector3(-1,0,0);
  }  
}
\end{lstlisting}

Por �ltimo, para realizar el salto, se utiliza una curva de Bezier con 3
puntos, de esta manera se consigue un movimiento m�s natural y atractivo
para el jugador. Para ello, se utiliza el c�digo de la clase
\texttt{Utils.cs} que se ha utilizado durante las pr�cticas de clase:

\begin{lstlisting}[caption={Salto del robot}]
/**
 * Realiza la accion de salto.
 * Segun el parametro lo realiza hacia arriba o hacia abajo.
 * Simula una curva de Bezier de 3 puntos para realizar un salto algo mas "natural"
*/
public void Jump(bool up = true){
  this.isJumping = true;
  this.isRotating = false;
  this.isMoving = false;
  
  //iniciamos los puntos de bezier con el primero y el ultimo
  bezierPts = new List<Vector3>();
  bezierPts.Add ( this.transform.position );  // Current position
  
  Vector3 _poi = GetJumpPoi (up);
  bezierPts.Add (this.transform.position+ ((_poi - this.transform.position) / 2 + 4*this.dirArriba));
  bezierPts.Add ( _poi );                   
  
  this.timeStart = Time.time;
  this.timeDuration = this.MVTO_DURATION;
  
}

/**
* Devuelve a partir de la posicion actual, de donde se este mirando, y del tipo de salto
* la posicion final del movimiento.
**/
private Vector3 GetJumpPoi(bool arriba = true){
  
  Vector3 aux = (arriba) ? this.dirArriba : -1*this.dirArriba;
  
  return aux + this.GetAvanzaPoi ();
}
\end{lstlisting}


\subsection{La l�gica}
\label{s4:suibsec:logica}

La clase \texttt{Robot.cs} es la encargada de realizar las comprobaciones
antes de realizar un movimiento por si no es posible. Para ello, almacena
de manera local la posici�n y rotaci�n actual del robot, y accede a la
definici�n del tablero de \texttt{GUI\_Layouyt.cs} para comprobar si el
movimiento es v�lido. El c�digo que realiza estas acciones es el siguiente:

\begin{lstlisting}[caption={L�gica del robot antes de realizar el
      movimiento.}]
using UnityEngine;
using System.Collections;

/**
 * Clase encargada de la gestion del robot.
 * Antes de realizar una accion comprueba si es posible.
 * Los movimientos los realiza a traves del componente Movement asociado al robot.
 */
public class Robot : MonoBehaviour {
  public static Robot S; //Singleton
  public Movement mvt; //Referencia al componente de movimiento.
  
  
  public int x, y, z; //Posicion actual del robot. Sirve para comrpobar si el siguiente movimiento es posible
  public int _x, _z; //Indica hacia donde esta mirando para poder saber hacia donde moverse
  

  
  public void Start(){
    S = this;
    this.mvt = (Movement)this.GetComponent<Movement>();
  }

  
  public void init(int posx, int posy, int posz, int rotx, int rotz){
    x = posx; y = posy; z = posz;
    _x = rotx; _z = rotz;
  }
  


  // Avanzar el robot hacia delante.
  public void moveUP(){
    //1.- Comprobar que se puede mover a la nueva posicion
    int newpos_x = x + _x;
    int newpos_z = z + _z;
    
    //1.1- esta dentro del tablero
    if(newpos_x <0 || newpos_z <0 ||
       newpos_x >= GUI_Layout.S.MAX_SIZE || newpos_z > GUI_Layout.S.MAX_SIZE)
       return;
    
    //1.2- existe la celda
    if (GUI_Layout.S.board_def[newpos_x, newpos_z].type == TileType.none)
       return;


    //1.3- esta a la misma altura
    if (y != GUI_Layout.S.board_def [newpos_x, newpos_z].height)
       return;

    //2.- Mover
    this.mvt.Avanza ();
    this.x = newpos_x;
    this.z = newpos_z;
    
  }


  // Saltar hacia delante
  public void jump(){
    //1.- Comprobar que se puede mover a la nueva posicion
    int newpos_x = x + _x;
    int newpos_z = z + _z;
    
    //1.1- esta dentro del tablero
    if(newpos_x <0 || newpos_z <0 ||
       newpos_x >= GUI_Layout.S.MAX_SIZE || newpos_z > GUI_Layout.S.MAX_SIZE)
       return;
		
    //1.2- existe la celda
    if (GUI_Layout.S.board_def[newpos_x, newpos_z].type == TileType.none)
       return;
		

       //1.3- esta a una altura de diferencia
       if (!(y == GUI_Layout.S.board_def [newpos_x, newpos_z].height+1 ||
           y == GUI_Layout.S.board_def [newpos_x, newpos_z].height-1 ))
	   return;
		
       //2.- Mover
       if (y > GUI_Layout.S.board_def [newpos_x, newpos_z].height)
         this.mvt.JumpDown ();
       else
         this.mvt.JumpUp ();
         
       this.x = newpos_x;
       this.z = newpos_z;
       this.y = GUI_Layout.S.board_def [newpos_x, newpos_z].height;
       
  }


  // Rotar hacia la derecha. Siempre se puede hacer
  public void rotateRight(){
    if (_x == 1 && _z == 0) {
      _x = 0;
      _z = -1;
    } else if (_x == -1 && _z == 0) {
      _x = 0;
      _z = 1;
    } else if (_x == 0 && _z == 1) {
      _x = 1;
      _z = 0;
    } else if (_x == 0 && _z == -1) {
      _x = -1;
      _z = 0;
    }
    
    this.mvt.RotateRight ();
  }
  

  // Rotar hacia la izquierda. Siempre se puede hacer
  public void rotateLeft(){
    if (_x == 1 && _z == 0) {
      _x = 0;
      _z = 1;
    }else if (_x == -1 && _z == 0) {
      _x = 0;
      _z = -1;
    }else if (_x == 0 && _z == 1) {
      _x = -1;
      _z = 0;
    }else if (_x == 0 && _z == -1) {
      _x = 1;
      _z = 0;
    }
    
    this.mvt.RotateLeft ();
  }
}
\end{lstlisting}
