\section{Caracter�sticas principales de la implementaci�n}
Las caracter�sticas principales de la implementaci�n del juego vienen
resumidas a continuaci�n:

\begin{itemize}
\item Se utilizan dos c�maras para componer la interfaz del juego. Cada
  c�mara muestra subespacio de la escena. La interfaz de cada c�mara se
  encuentra gestionada por su propio script.
\item Dise�o de niveles a trav�s de una estructura xml. Permite crear y
  modificar niveles de manera sencilla.
\item Utilizaci�n de sprites para la creaci�n de botones.
\item utilizaci�n de distintos materiales desde el c�digo que permite
  cambiar el fondo de los botones y algunos aspectos.
\item Utilizaci�n de curvas de Bezier para el movimiento de salto del
  robot, produciendo un movimiento m�s natural. Los movimientos se
  implementan en un componente aparte permitiendo su reutilizaci�n,
  mientras que la l�gica de los movimientos (saber si se puede mover o no),
  la implementa el componente Robot.
\item Control de errores antes de ejecutar una acci�n. Si el robot no es
  capaz de ejecutar una acci�n, contin�a ejecutando la siguiente.
\item Creaci�n de un sistema basado en una pila para simular la
  reproducci�n de los comandos. Permite generar una traza de la ejecuci�n
  (mostrada en formato visual mediante los colores del fondo de las
  acciones), y simular llamadas recursivas y llamadas anidadas.
\item Utilizaci�n de dos escenas: la escena de men� y la escena de juego.
\item La escena de juego es la misma para todos los niveles, lo que se
  cambia es el fichero xml desde el que leer el mapa.
\item Para la gesti�n entre el men� y la escena del juego, as� como para
  almacenar los niveles superados se utiliza una clase est�tica que
  permanece instanciada durante toda la ejecuci�n.
\item La creaci�n de los paneles donde insertar los comandos se realiza de
  forma procedural, permitiendo crear tantos paneles como sean necesarios,
  as� como elegir el tama�o de cada uno de los paneles.
\end{itemize}
